% Latex Style for CMP PGR DAY 2009.
%
% Revision 1.0
% Feb. 12 2009.
%
% Barry-John Theobald, University of East Anglia, Norwich, UK

\documentclass{cmppgr}

% For urls
\usepackage{hyperref}

% For programming font
\usepackage{courier}

\title{Computer Games: ARGame}
\name{James Rogers, 100062949}
\institution{
	School of Computing Sciences, University of East Anglia, UK
}
\email{}



\begin{document}
\maketitle

\begin{abstract}

Abstract.

\end{abstract}

% % % % % % % % % % % % % % % % 
% % %		Introduction	   % % %
% % % % % % % % % % % % % % % % 
\section{Introduction}

% % % % % % % % % % % % % % % % 
% % %		Requirements   	% % %
% % % % % % % % % % % % % % % % 
\section{Requirements}

\subsection{Specification}
\subsection{MoSCoW}

% % % % % % % % % % % % % % % % 
% % %			Design			 % % %
% % % % % % % % % % % % % % % % 
\section{Design}

\subsection{IID}
\subsubsection{Version 0.1}


\subsection{Technologies}
\subsection{Inital Architecture}
\subsection{Final Architecture}
\subsection{Finite State Machine}


% % % % % % % % % % % % % % % % 
% % %	Implementation	   % % %
% % % % % % % % % % % % % % % % 
\section{Implementation}

\subsection{Graphics}
\subsubsection{Drawing}
\subsubsection{Model Loading}
Model loading is performed using the \texttt{ModeLoader} class. It provides an interface of static methods which take a file path to the model file as their parameter, then return an instance of the \texttt{Model} class with the data loaded inside it. 

The Assimp\footnote{\url{www.assimp.org}} model loading library was used to read 3D mesh and material data from Wavefront .obj formats. The Assimp library is written in C++, therefore a class, \texttt{AssimpModelLoader} was implemented in C++ to interact with the Assimp library, interpret and temporarily store it's output. The Swift \texttt{ModelLoader} class interacts with \texttt{AssimpModelLoader} through a bridging header; which consists of C functions that Swift can interact with to retrieve data from \texttt{AssimpModelLoader}. 

\subsubsection{Photogrametery}


\subsection{Collisions}
\subsubsection{Collision Detection}
\subsubsection{Collision Resolution}

\subsection{Augmented Reality}
\subsubsection{Marker Tracking}
\subsubsection{Smoothing}
\subsubsection{Illumination Model}

% % % % % % % % % % % % % % % % 
% % %			Testing	  		 % % %
% % % % % % % % % % % % % % % % 
\section{Testing}

\subsection{Unit Testing}

\subsection{Intergration Testing}

% % % % % % % % % % % % % % % % 
% % %		Conclusion	 	  % % %
% % % % % % % % % % % % % % % % 
\section{Conclusion}



% % % % % % % % % % % % % % % % 
% % %		Bibliography	   % % %
% % % % % % % % % % % % % % % % 

%\bibliographystyle{apalike}
\bibliographystyle{agsm}
%\bibliographystyle{abbrvnat}



\end{document}
